\documentclass[12pt, titlepage]{article}

\usepackage{color}
\usepackage{xspace}
\usepackage{longtable}
\usepackage{multirow}
\usepackage{alltt}
\usepackage{proof}
\usepackage{hyperref}

\newcommand{\makered}[1]{\textcolor{red}{#1}}
\newcommand{\redbold}[1]{\makered{\textbf{#1}}}
\newcommand{\note}[1]{\redbold{[#1]}}
\newcommand{\ignore}[1]{}
\newcommand{\missing}{\redbold{TBD}}


\title{Towards a Tool for Verified Hardware Construction}
\author{Jared Roesch,  \\
    Department of Computer Science, \\
    University of California Santa Barbara. \\
    }

\date{\today} 
% \date{\today} date could be today 
% \date{25.12.00} or be a certain date
% \date{ } or there is no date 
\begin{document}
% Hint: \title{what ever}, \author{who care} and \date{when ever} could stand 
% before or after the \begin{document} command 
% BUT the \maketitle command MUST come AFTER the \begin{document} command! 
\maketitle

\begin{abstract}
Designing secure hardware is an active area of research that has seen much growth in recent years.
Hardware is now more complicated then ever but even if we state and prove high level security 
properties about our designs we have no way to demonstrate a correspondence between the thing
we formally discuss and the thing that we eventually design. Unfortunately the tools used to construct hardware
have not fundamentally changed in decades. We present a high level overview of the state of the art with regards to these 
issues, and sketch a plan of how to adapt these techniques in order to build a toolchain for
the formal verification of hardware synthesis.
\end{abstract}

%\tableofcontents % create a table of contens 

\section{Introduction}

Hardware security is becoming and incresingly important topic both as a research problem and a pragmatic issue effecting 
day to day life. Hardware is multiplying around us and proliferating through all parts of life. We are seeing more and more
devices spring into existance to manage everything from avionics, pace makers, to your thermostat. One of the problems is
that we are giving these devices more and more control over critical systems that govern the safety and lives of humans. At
the same time we also have issues with mobile security, cloud computing, and more.

Traditionally many of these problems have only been thought about at the software level and much of the related work has
been about verifing correctness and security properties about the systems running on top. Ideally we want to be able an
run entire stack of critical software; a verified application, running on a verified operating system. This picture
unforuntately doesn't factor in vulnerabilities at the hardware level. Often times the hardware's security has been
an afterthought, but there has been multiple recent examples of attacks that leveraged hardware level exploits in order
to compromise systems(\makered{cite here}).

The missing part of this equation then becomes the verification of hardware. Hardware verification is not necessarily a trivial
problem. Although hardware construction has similarities with design and verification of software, there are 
key issues that must be solved. There has been (\makered{insert citations}) in verifiying simple hardware description, and properties ofhardware. (More \makered{limitations here}) The goal then is to build a tool focused on hardware construction that exposes
more powerful tools to the engineer who designs hardware. 

(\makered{incorporate these ideas})
Unforunately previous approaches present a DSL in a 
Theorem Proving environment. There is a endeavor to free the hardware designer to be able to both build
hardware and prove properties.

\section{Background}
There has been distinct work in each area of secure computer arhcitectures, and 
hardware description languages, and theorem proving. We survey the important highlights from each
that are necessary to understanding our approach.
\makered{more text here probably}

\subsection{Secure Architecture}
\begin{itemize}
    \item Ruby's work, Tim's on execution leases, Appel 
    \item All high level properties that are loosely reasoned about
    \item want flexibility in design, parametrizable modules (show example from GitHub here)
    \item Missing rigorous formal analysis
    \item When they have analysis it suffers from being Pen and Paper, and being abstracted from the actual work.
\end{itemize}

In \cite{lee05} they describe “secret-protected” architecture that enables the protection of critical secrets for users in an online environment. They argue that their architecture requires only a few extra features to be built into a general purpose microprocessor to protect secrets and related computations. They do this with a small amount of extra hardware. They are able to decouple user secrets from the devices, avoid symmetric master keys, and avoid doing factory-installed device secrets. Their approach can be divided into a few core features the Key Chain, Trusted Software Module, Concealed Execution, Hardware Extensions, Secure I/O, and OS support. \cite{exec-leases} \cite{PicoCoq2013:POPL}.

\cite{blackboxed} \cite{languagebased} \cite{Chlipala:2007:CTC:1250734.1250742}

\makered{work some critique in here: The evaluation could use some work. There is a written description of both their security and performance evalution, but neither has explicit numbers. It seems that although Ruby talks about performance vs. secu- rity tradeoffs in her book, she is not completely clear about the ones made here. As well they are lacking any sort of formal proof of their security properties. This may not be important to the architecture community, but it seems questionable at best to make claims about the security of a system at a high level without actually formally verifiying the properties hold in practice. The performance evaluation also seems to be very high level, and ignores important details, for example they dismiss the performance impact on the overall system without presenting precise empirical evaluation of it. They do evaluate the instruction load of their hardware crypto but do not compare it aganist anything else in the space.}


In \cite{languagebased} they described ways to add a type system to a small imperative language in order to mitgate the leakage of timing information.
Ideally these hardware concerns shouldn't leak beyong the hardware level, and they grappel with this in their analysis of the technique. 

\makered{put this somewhere: It would be desireable to adapt their technique from \cite{blackboxed} and use our technique to help prove that correctness of the approach.}

\subsection{Hardware Description Languages}

Hardware description languages are software tools for synthesizing hardware from a high level programmatic description.
A hardware description language is ideally a precise, formal description of an electronic circuit that 
makes it easy to do automated analysis, simulation, and testing for a circuit. In reality they have
various weaknesses and failings \subsection{Type Theory}
Type theory is a foundational framework for mathematics. It was an approach that evolved out of Russel's \cite{Principia}
as an approach to a consistent foundational system for mathematics. The early 20th century was spent developing type theory
as a proof system. The key insight was the Curry-Howard Isomorphism, which roughly states that types can be associated with
propositions and proofs with programs. This insight was adapted and it became apparent that typed variants of the lambda
calculus can be used as proof systems, the most famous of which is Martin-Lof type theory. You can then view types in
this typed lambda calculus as various features of constructive logic.

\makered{TODO: clean this section up with typeset examples and such} \\

If we begin with just an implicational fragment of logic we can use the simply typed lambda calculus which only has a
single type constructor, the function type. This corresponds to implication. We can now view the typing judgment $a : A$
as statement that a is a proof of A, and we can interepret the typing judgment $f : A -> B$ as a computational implication.
At the type level it allows us to produce a B given an A, and computationally it gives us a method for transforming a program
of A into a program of B. If we remember that program's corresond to proofs we now have a computational method for proving.

We can of course extend our type system with richer features. We will quickly survey the important additions to the Simply
Typed Lambda Calculus that are needed to turn it into a dependently typed lambda calculus. The most important feature
is the idea of Pi Types. Simple type systems can be enriched with many features such as products, sums, so on. There
have been innovations in type system construction, but depdently typed calculi in the style of the CoC or MLTT are
the most powerful. 

Traditonally we maintain a phase distinction where types and values leave in seperate universes. Depdent types
are about erasing this distinction. Types can be made more expressive by a series of additions. First if we
allow quantifers we can now abstract over types with type lambdas. We can then introduce kinds which allow us
to specify type arities. We can then introduce kind polymorphism, and many other features. Each of these systems become
more and more complex because they have to maintain constant distinction between each universe of values, kinds, sorts.
Depdent types simply introduce the idea of a Pi Type of dependent function space. This type system generalization is the 
idea that we can make everything much simpler by creating a stratfied hierarchy of types, and combining this Pi Types.
We then can collapse our language into a simple set of pseudo terms, and typing rules \makered{examples can be found}.

The idea of Pi Types can be simply viewed as a generalization of the traditional function type. The generalization
allows the codomain to depdend on the domain of the function. Given $B : P -> A; \Pi (x : A). B(x)$.

\cite{HoTTbook} \cite{Pierce:TypeSystems}

\subsection{Automated Theorem Proving}
High level talking points: \\
\begin{itemize}
    \item dependent types
    \item equal to a fragment of inutionistic logic
    \item provides tools for automated proof search
    \item specification and implementation are one and the same
    \item can do program extraction
    \item executable 
\end{itemize}

Automated Theorem Proving has been an area of active research for many decades. It is apparent that the
verification of general properties is desireable to both high assurance software engineers, researches, and
mathmaticians. There have been various approaches to Automated Theorem Proving in the past ACL2, NuPRL, LF,
LCF, Coq, Agda. The ones based on dependent type theories seem the most promising for various reasons 
(computational, program extraction, proof carrying code, ect). The cutting edge of type theory has been
an evolutions of the work done my Per Martin-Lof and others.

Automated Theorem Proving has been applied in many areas from the NTSB, NASA, JPL, and a vareity of other places.

The problem is that very often the specification is seperate from the implementation and suffers the common
problems of pen and paper proof. In many cases Embedded Domain Specific Languages (EDSLs)
have been used as a way to gain the power of a proof assitant and also write proof carrying code. We want
to leverage the ideas here, but ideally built a seperate front-end that exposes the full power of proving,
but also allows one to provide both an usuable interface that also generates useful proof properties for
free \cite{Ricketts:2014}. \cite{chlipala2011certified} \cite{Pierce:SF}
\\
\makered{NOTE: Work in (For now I'm using a proof environment called Idris for prototyping verified Hardware construction.)}

\subsection{Verified Hardware Construction}
There has been work on creating tools for doing designing secure hardware such as Caisson and SAFE but in both cases the
compiler assumes and enforces a standardized security policy. \makered{elaborate} \cite{Li:2011:CHD:1993498.1993512}
There also has been work on specialized systems like Cryptol that make simplifying assumptions allowing certain gaurentees
and properties to hold. There has been some related work in building higher level design languages such as Lava, and Bluespec
that allow the hardware designer to abstract over properties such as explicit timing information. The most closely related
work has been Fe-Si \cite{fesi} \cite{Brady_constructingcorrect} \cite{cryptol2010}

\section{Extensions}

\begin{itemize}
    \item consistent design
    \item demonstrate examples like verification of overflow
    \item demonstrate bounded input sizes
\end{itemize}

\begin{itemize}
    \item consistent design
    \item syntactic/semantic seperation unlike verilog behavioral/structural
    \item allow for some properties to be synthesized and proved automaticallly
    \item verify synthesize and semantics of HDL so that any well formed HDL program has certain properties (ala Coq EDSLs)
    \item \makered{find citation for UCSD PLDI '14 paper} \cite{Ricketts:2014}
\end{itemize}

Key weaknesses of previous approaches to verification:
\begin{itemize}
    \item toy examples
    \item no automatic property generation/proof satisfaction
    \item 
\end{itemize}

\makered{still prototyping the things I want to discuss here}
\makered{The goal is to take everything that has been described and synthesize the ways in which we can build a tool.}

\section{Conclusion}

There has been a lot of work poured into constructing architectures that exhibit desired security properites. Unfortunately there hasn't been
much work on the implementation and verification of these architectures so although on paper we are able to reason about the behavior of the
system we don't have many guarentees about the behavior of our actual implementation. This is a failing of the tools used by hardware
designers. We present a blueprint of how to leverage work done in the programming languages community and leverage automated theorem provers
and dependent types as tools for specifying and verifying properties of hardware designs. We leave the more diffcult problem of designing and
verifying hardware designs to future work, the first goal is to build a tool that is powerful and robust enough to actually port real designs
to it.

\bibliography{paper}
\bibliographystyle{plain}

\end{document}
