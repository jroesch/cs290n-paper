\documentclass[12pt, titlepage]{article}

\usepackage{color}
\usepackage{xspace}
\usepackage{longtable}
\usepackage{multirow}
\usepackage{alltt}
\usepackage{proof}
\usepackage{hyperref}

\newcommand{\makered}[1]{\textcolor{red}{#1}}
\newcommand{\redbold}[1]{\makered{\textbf{#1}}}
\newcommand{\note}[1]{\redbold{[#1]}}
\newcommand{\ignore}[1]{}
\newcommand{\missing}{\redbold{TBD}}


\title{Towards a Tool for Verified Hardware Construction}
\author{Jared Roesch,  \\
    Department of Computer Science, \\
    University of California Santa Barbara. \\
    }

\date{\today} 
% \date{\today} date could be today 
% \date{25.12.00} or be a certain date
% \date{ } or there is no date 
\begin{document}
% Hint: \title{what ever}, \author{who care} and \date{when ever} could stand 
% before or after the \begin{document} command 
% BUT the \maketitle command MUST come AFTER the \begin{document} command! 
\maketitle

\begin{abstract}
Designing secure hardware is an active area of research that has seen much growth in recent years.
Hardware is now more complicated then ever but even if we state and prove; high level security 
properties about our designs, we have no way to demonstrate a correspondence between the model
we formally discuss and the implementation that we eventually design. In the face of changing
demands the tools used to construct hardware have not fundamentally changed in decades. 
We present a high level overview of the state of the art with regards to these 
issues, and sketch a plan of how to adapt these techniques in order to build a tool chain for
the formal verification of hardware synthesis.
\end{abstract}

%\tableofcontents % create a table of contens 

\section{Introduction}

Hardware security is becoming and increasingly important topic both as a research problem and one that has an effect on
day to day life. Hardware is multiplying around us and proliferating through all parts of life. We are seeing more and more
devices spring into existence to manage everything from avionics, pace makers, to your thermostat. An issue then
becomes, that as we give these devices more and more control, we are more and more interested in moderating their control. This is
because control of these devices isn't just the device or its contents as it has historically been, but access to sensitive personal
control, for example an exploited iPhone may have greater implications when it can open your doors, or command household electronics.
Households could be made vulnerable to "malfunction" whether malicious or not. As well there are issues to be had with mobile security, as 
smart phones play a more and more important role in in daily life. It goes hand in hand with cloud computing, which is playing out as a nice
dual to the rise of smart phones, more and more computation has been moved centrally to the "cloud" and it introduces central points of failure
if they are compromised. No matter the platform it is necessary to begin ensuring the integrity of our software from the top down. Even if we
introduce notions of verified software it is meaningless in the face of exploitable hardware.

Traditionally the specification and verification of security properties has only been done at an architectural 
level, by either design or by a baked in policy, and enforcement . A lot of work has been done about
properties been about verifying correctness and security properties about the systems running on top. The ideal presented 
in older works was to be able to run an entire stack of critical software; a verified application, running on a verified operating system. 
This picture unfortunately doesn't factor in vulnerabilities at the hardware level. Often times the hardware's security has been
an afterthought, but there has been many recent examples of attacks that leveraged hardware level exploits in order
to compromise systems. 

The missing part of this equation then becomes the verification of hardware. Hardware verification is not necessarily a trivial
problem. Although hardware construction has similarities with design and verification of software, there are 
key issues that must be solved. There has been success in verifying simple hardware description, 
and properties of the designs. The goal is to take the kernel of an idea and flesh it out into
a tool focused on hardware construction that exposes very powerful tools to the engineer who designs 
hardware.

Unfortunately the state-of-the-art \cite{fesi} only presents a limited domain specific language
in a theorem proving environment, providing very little support to to a hardware designer who wants to be 
able to write designs and then prove properties. We believe that it is possible to do better providing a 
pragmatic and useful tool that allows us to leverage the power of some properties for free, as well
as the ability to add more via the proof system, enabling hardware design. 

\section{Background}
There has been distinct work in each area of secure computer architectures, and 
hardware description languages, and theorem proving. We survey the important highlights from each
that are necessary to understanding our approach.

\subsection{Secure Architecture}

In \cite{lee05} they describe “secret-protected” architecture that enables the protection of critical secrets for users in an online environment. They argue that their architecture requires only a few extra features to be built into a general purpose microprocessor to protect secrets and related computations. They do this with a small amount of extra hardware. They are able to decouple user secrets from the devices, avoid symmetric master keys, and avoid doing factory-installed device secrets. Their approach can be divided into a few core features the Key Chain, Trusted Software Module, Concealed Execution, Hardware Extensions, Secure I/O, and OS support. \cite{PicoCoq2013:POPL}. It seems that although Lee talks about performance vs. Security trade offs in her book, though their work is not clear about which trade offs they made. They also lack formal proof of their security properties. This may not be important to the architecture community, but it seems questionable at best to make claims about the security of a system at a high level without actually formally verifying the properties hold in practice.

In \cite{blackboxed} \cite{languagebased} they discuss mechanisms for architectural and language support for mitigating the
leakage of information through side channels. They take the approach of inserting time stops in order to prevent implicit
leaks from control flow structures. 

In \cite{languagebased} they described ways to add a type system to a small imperative language in order to mitigate the leakage of timing information.
Ideally these hardware concerns shouldn't leak beyond the hardware level, and they grapple with this in their analysis of the technique. 

In \cite{exec-leases} they also detail an information flow architecture base upon their GLIFT technique. The critical idea of the work is that
we can provide secure leases that allow for a caller to bound the privileges both in terms of information and time for the lessee. 

Having identified some of the weaknesses in the implementations of secure architectures we would 
ideally like to compliment them by providing a tool that enables the following criteria:
\begin{itemize}
    \item state high and low level security properties
    \item reason about abstract specification
    \item correspondence between specification and hardware
    \item missing rigorous formal analysis
\end{itemize}


\subsection{Hardware Description Languages}

Hardware description languages are software tools for synthesizing hardware from a high level programmatic description.
A hardware description language is ideally a precise, formal description of an electronic circuit that 
makes it easy to do automated analysis, simulation, and testing of a circuit. In reality they have
various weaknesses and failings that mostly result from the weaknesses introduced by current tooling. 
We would ideally like to make it easier to state precise specifications of hardware, and verify that
the implementations matches said specification.

\subsection{Type Theory}
Type theory is a foundational framework for mathematics. It was an approach that evolved out of Russel's work on setting
mathematics on a consistent foundational system for mathematics. The early 20th century was spent developing type theory
as a proof system. The key insight was the Curry-Howard Isomorphism, which roughly states that types can be associated with
propositions and proofs with programs. This insight was adapted and it became apparent that typed variants of the lambda
calculus can be used as proof systems, the most widely used of which is Martin-Lof type theory. You can then view types in
this typed lambda calculus as various features of constructive logic.

If we begin with just an implicational fragment of logic we can use the simply typed lambda calculus which only has a
single type constructor, the function type. This corresponds to implication. We can now view the typing judgment $a : A$
as statement that a is a proof of A, and we can interpret the typing judgment $f : A -> B$ as a computational implication.
At the type level it allows us to produce a B given an A, and computationally it gives us a method for transforming a program
of A into a program of B. If we remember that program's correspond to proofs we now have a computational method for proving.

We can of course extend our type system with richer features. We will quickly survey important additions to the Simply
Typed Lambda Calculus that are needed to turn it into a dependently typed lambda calculus. The key idea is the erasure
of phase distinction via the inclusion of a stratified type hierarchy, and Pi types.

Traditionally we maintain a phase distinction where types and values live in separate universes. Dependent types
are about easing this distinction. Type systems can usually be made more expressive by a series of additions.
The simplest of which is the introduction of quantifiers which allow us to abstract over types with type lambdas. 
We can then introduce kinds (the types of types), and kind polymorphism which allow us to specify type shapes or (type arities).
We can continue to enrich our kind and type system, eventually reflecting types as kinds, and values as types, 
but these systems are actually more complicated then a dependently typed calculi. Each of these refines become
more and more complex because it requires the language to maintain a discipline between values, kinds, sorts.
Dependent types simply introduce the idea of a Pi Type of dependent function space. This type system generalization is the 
idea that we can make everything much simpler by creating a stratified hierarchy of types, and adding Pi and Inductive types.
We then can collapse our language into a simple set of pseudo terms, and typing rules which are easy to verify for correctness.

We can then ignore all the complexity introduced by types systems such as System F and generalizations of it. Instead we can 
focus on a simple language. Usually a dependently typed core only requires a few forms abstraction, variables, function application,
and Pi Types. This is simply the lambda calculus that everyone knows enriched with Pi types. It is also useful to extend the calculus
with inductive types, or recursive data types that may be parametrized by the values they are constructed with.

Pi Types can be simply viewed as a generalization of the traditional function type. The generalization
allows the type of codomain to depend on value of the domain of the function. Given $B : P -> A; \Pi (x : A). B(x)$. This
simple extension actually gives rise to a lot of power and will allow us to encode much of mathematics in type theory.

This is of course a very brief overview and more information can be found in \cite{Pierce:TypeSystems} \cite{martinlof} and \cite{HoTTbook}.

\subsection{Theorem Proving}

Automated Theorem Proving has been an area of active research for many decades. It is apparent that the
verification of general properties is desirable to both high assurance software engineers, researches, and
mathematicians. There have been various approaches to Automated Theorem Proving in the past ACL2, NuPRL, LF,
LCF, Coq, Agda. The ones based on dependent type theories seem the most promising for various reasons 
(computational, program extraction, proof carrying code, ect). Automated Theorem Proving has been applied in
many areas from the NTSB, NASA, JPL, and a variety of other places.

The current best way to leverage the proving power of Type Theory are Automated or Interactive Theorem Proving
environments. These allow for the automatic satisfaction of some properties, and interactive proving of others.
The state-of-the-art theorem proving environments in use for software verification are Coq, Agda, and Idris 
all of which provide the benefits of:
\begin{itemize}
    \item dependent types (equivalent to a fragment of isolationistic logic)
    \item provides tools for automated proof search
    \item specification and implementation are one and the same
    \item executable "proofs" (i.e program extraction)
\end{itemize}

These are all based on versions of intensional type theory. Coq is base on
on Coquand and Huet's Theory of Constructions, and Agda is an evolution 
Martin-Lof Type Theory \cite{martinlof}.

Automated Theorem Proving has been an area of active research for many decades. It is apparent that the
verification of general properties is desirable to both high assurance software engineers, researches, and
mathematicians. There have been various approaches to Automated Theorem Proving in the past ACL2, NuPRL, LF,
LCF, Coq, Agda. The ones based on dependent type theories seem the most promising for various reasons 
(computational, program extraction, proof carrying code, ect). Th current state-of-art is either based
on Coquand and Huet's Theory of Constructions or the work done on Martin-Lof Type Theory \cite{martinlof}.
Automated Theorem Proving has been applied in many areas from the NTSB, NASA, JPL, and a variety of other places.

It appears that ideas from ATP could be very useful in the domain of hardware specification. Most
specification is performed separately from the implementation and suffers the common
problems of pen and paper proof. In many cases Embedded Domain Specific Languages (EDSLs)
have been used as a way to gain the power of a proof assistant and also write proof carrying code\cite{Ricketts:2014}\cite{fesi}. We want
to leverage the ideas here, but ideally built a separate front-end that exposes the full power of proving,
but also allows one to provide both an usable interface that also generates useful proof properties for
free \cite{Ricketts:2014}. \cite{chlipala2011certified} \cite{Pierce:SF}

There is a prototype implementation of ideas discussed in the final section available. It is currently
written in Idris one of the more practical dependently typed programming language.

\subsection{Verified Hardware Construction}

There has been work on creating tools for aiding in the design of secure hardware such as Caisson which presents a HDL 
language that has a preselected static information flow policy. This provides a information flow policy by baking in 
a static analysis to the compiler. We would like to be able to easily experiment with different different security 
properties and be able generally state arbitrary security properties of the hardware. We can easily discharge these
proof assumptions by a mixture of proof automation and interactive proving. SAFE takes a similar approach to the
compiler assumes and enforces a standardized security policy. This is also the case with Caisson a gate level 
approach that is baked in to the HDL compiler \cite{Li:2011:CHD:1993498.1993512}.

There also has been work on specialized systems like Cryptol that make simplifying assumptions allowing certain guarantees
and properties to hold \cite{cryptol2010}. Cryptol is able to use tools like SMT solvers and simple sized type system in
order to prove properties about cryptography implementations. They are able to make these simplifications be identifying 
ways in which traditional cryptography hardware is constructed and designing their language accordingly.

 There has been some related work in building higher level design languages such as Lava, and Bluespec
that allow the hardware designer to abstract over properties such as explicit timing information. The most closely related
work has been Fe-Si \cite{fesi} which can viewed as a deterministic version of BlueSpec. The key insights to take away from
the Fe-Si work is the description of how to use the meta languages type system for great effect, as well as inight into
the verificatin of HDL compiler passes. Adam Chilpala has done a lot work on describing novel ways in which to verify different
aspects of programming languages \cite{Chlipala:2007:CTC:1250734.1250742}\cite{chlipala2011certified}

There also has been some work from the dependently typed programming language community sketching ways in which dependent types
can be made useful in hardware design\cite{Brady_constructingcorrect}. The authors only scratch the surface of the issues that
occur in real hardware design and take a tutorial like approach to describing how to leverage dependent types to assist in 
design correctness, demonstrating very simple combinational designs like adders.

\section{Extensions}

Our general goals are to leverage ideas from program analysis, type systems and
apply them to a hardware construction language. We motivate our desire to replace existing HDLs for the
reasons we listed in the previous section. Primarily it is about pursuing a more consistent design, that
allows us to have both concise hardware descriptions, and write proofs about said specifications. We
also want to equip the surface language with the ability to demand certain properties and proofs for
free. For example we want to provide guarantees of properties such as overflow, prop wiring by construction,
and so on, but leave the ability to establish higher level proof properties like the information
flow expressed by default in Caisson's compiler \cite{Li:2011:CHD:1993498.1993512}.

\begin{itemize}
    \item allow for some properties to be synthesized and proved automatically
    \item verify synthesize and semantics of HDL so that any well formed HDL program has certain properties similar to \cite{Ricketts:2014}
\end{itemize}

Most previous approaches have not been practical enough to use for building real designs. Fe-Si is as close to
complete but is still unfortunately just a toy. The DSL approach has important limitations and requires
programming in an awkward monadic style inside of the theorem proof system Coq. Ideally we would like to take the 
simple core described in the Fe-Si work, but extend it in a few key ways. First we would like to add a surface language
that will give us the ability to add more complicated features than the limited Fe-Si. We will also like to make modules first class
entities that express information about the functionality of the circuit as well as 

We have experimented with a few different designs, and am still considering which is the best way forward.
So far we have not made a decision on which is best, but have noticed some important trade offs in the process
of implementing 

\section{Conclusion}

There has been a lot of work poured into constructing architectures that exhibit desired security properties. Unfortunately there hasn't been
much work on the implementation and verification of these architectures so although on paper we are able to reason about the behavior of the
system we don't have many guarantees about the behavior of our actual implementation. This is a failing of the tools used by hardware
designers. We present a blueprint of how to leverage work done in the programming languages community and leverage automated theorem provers
and dependent types as tools for specifying and verifying properties of hardware designs. We leave the more difficult problem of designing and
verifying hardware designs to future work, the first goal is to build a tool that is powerful and robust enough to actually port real designs
to it.

\bibliography{paper}
\bibliographystyle{plain}

\end{document}
