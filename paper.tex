\documentclass[12pt, titlepage]{article}
% font size could be 10pt (default), 11pt or 12 pt
% paper size coulde be letterpaper (default), legalpaper, executivepaper,
% a4paper, a5paper or b5paper
% side coulde be oneside (default) or twoside 
% columns coulde be onecolumn (default) or twocolumn
% graphics coulde be final (default) or draft 
%
% titlepage coulde be notitlepage (default) or titlepage which 
% makes an extra page for title 
% 
% paper alignment coulde be portrait (default) or landscape 
%
% equations coulde be 
%   default number of the equation on the rigth and equation centered 
%   leqno number on the left and equation centered 
%   fleqn number on the rigth and  equation on the left side
%    

\title{Designing Provably Correct Secure Architectures}
\author{Jared Roesch,  \\
    Department of Computer Science, \\
    University of California Santa Barbara. \\
    }

\date{\today} 
% \date{\today} date could be today 
% \date{25.12.00} or be a certain date
% \date{ } or there is no date 
\begin{document}
% Hint: \title{what ever}, \author{who care} and \date{when ever} could stand 
% before or after the \begin{document} command 
% BUT the \maketitle command MUST come AFTER the \begin{document} command! 
\maketitle

\begin{abstract}
Secure hardware is obviously a very active interesting area of research.
People are working hard on desinging architectures that have properties
that are desireable for a vareity of reasons.
How do we reason about \cite{exec-leases}
- restrictions on computation
- static reasoning
- instead of encoding dynamic checks or assuming that

\end{abstract}

%\tableofcontents % create a table of contens 

\section{Introduction}

Hardware security is becoming and incresingly important topic both as a research problem and a pragmatic issue effecting 
day to day life. Hardware is multiplying around us and proliferating through all parts of life. We are seeing more and more
devices spring into existance to manage everything from avionics, pace makers, or your thermostat. One of the problems is
that we are giving these devices more and more control over critical systems that govern the safety and lives of humans. At
the same time we also have issues with mobile security, cloud computing, and more.

Traditionally many of these problems have only been thought about at the software level and much of the related work has
been about verifing correctness and security properties about the system running on top.

The problem is that if we want to truly verify correctness properties we must verify each piece of the stack from top to 
bottom.

The missing part of this equation then becomes the verification of hardware. Obviously there has been other work here 
(insert citations) in verifiying simple hardware description, and properties of hardware. What we want to do is introduce
more powerful tools to the hardware developer who actually describes and designs hardware. Unforunately previous approaches
present a DSL in a Theorem Proving environment. There is a endeavor to free the hardware designer to be able to both build
hardware and prove properties.

\section{Background}
- Secure Hardware
- Proving
- DTs
- ect

\section{Secure Architecture Design}

\section{Verified Hardware Construction}

\section{Extensions}

The goal is to take everything that has been described and combine them into something that is uni

\bibliography{paper}
\bibliographystyle{plain}

\end{document}
